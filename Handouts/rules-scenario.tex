%%%%%
%%
%%%%%

\documentclass[sheet]{TMFHope}

\usepackage{hyperref}
\usepackage{multicol}
\usepackage{ltablex}
\usepackage{tabularx}
\renewcommand{\tabularxcolumn}[1]{m{#1}}
\setlength{\columnsep}{1cm}

%% document-wide tweaks
\interlinepenalty10000
\setstretch{1}
\def\mytype{Rules and Scenario}
\lfoot{}\rfoot{}
\parindent0pt

\begin{document}

%% layout for cover page
\thispagestyle{empty}
\parskip0pt

%% title box
\begin{center}\LARGE\bf\begin{tabular}{|c|}
  \hline \gamename\\ \gamedate\\ Rules and Scenario\\ \hline
\end{tabular}\end{center}

\vfill\vfill

%% player side of the GM/player contract
The following are the rules for {\em\gamename}, a real-time, discord-based, roleplaying game sponsored by the Luminary Roleplay Society.\\

You are responsible for knowing and following these rules. It is through the constraints created by these rules and the game mechanics that interesting character to character interactions are made possible. Many of these rules are nigh-impossible to enforce and rely upon the honor system.  Do not cheat.  Do not abuse loopholes.  Play fair.  Be your own harshest critic. To do otherwise will cheat both yourself and your fellow players out of the intended experience.

\vfill

%% GM side of the GM/player contract
The \textbf{gamemasters} (\textbf{GMs}) run the game.  If you have any problems or questions concerning the game, contact a GM.  The GMs are available before game to clarify any questions you may have about the rules and game. Th GMs will also be available during game, but the more time and attention you put toward understanding the rules and mechanics ahead of time, the less overwhelmed you will likely feel, and the more you are likely to enjoy the game.  Rulings the GMs make are final.  They may violate the letter of the rules to preserve the spirit.  The GMs promise to be as fair and reasonable as possible. Neither they nor these rules are perfect, so we ask for your understanding and flexibility.

\vfill

%% have fun
This game is intended to be fun.  Getting into character, roleplaying, being dramatic, and competing in character can all increase the fun of the game.  Do not take the game too seriously though.  Even if your character is ``losing'', keep a good attitude as a player.  When the game is over, the real winners are the players with the best stories.

\vfill

%% disclaimer and copyright
%% author list auto-generated from Lists/gm-LIST.tex
This game is a work of fiction.  Although it may refer to things in the real world, it does so only for the sake of the scenario.  It does not represent the opinions of the GMs or the author. These rules are modifications of those used in previous games.  This game and all materials thereof are copyright 2020 by Acata Felton and Olivia Montoya.\\

Thanks is due to Olivia Montoya for writing the Discord Bot that makes this game possible. ``TMF Hope'' is written by Acata Felton (she/her; char.felton@gmail.com) and Olivia Montoya. Please ask for permission before running the game, but also please do take the game and run it for your friends! Playtesting was enabled by the Luminary Roleplay Society, a San Francisco Bay Area 501(c)(3) non-profit that promotes LARPing in the Bay area and at conventions across the United States. If you would like to donate to the upkeep of the discord bot, you can do so here: \href{paypal.me/oliviam11}{paypal.me/oliviam11} or here: \href{http://ko-fi.com/metaparadox}{http://ko-fi.com/metaparadox}. If you would like to donate to ongoing LRS activities, you can donate here: \href{https://www.luminaryroleplaysociety.com/donations/}{https://www.luminaryroleplaysociety.com/donations/}. The recommended donation is \$5 to one or the other, or both, source(s).

\vfill\vfill

\begin{center}\bf
  BROUGHT TO YOU BY THE LUMINARY ROLEPLAY SOCIETY
\end{center}

\vfill

\clearpage

%% layout for Table of Contents page
\thispagestyle{empty}
\tableofcontents

\clearpage

%% layout for main body of rules
\setcounter{page}{1}
\parskip5pt


\section{Introduction and Logistics}

\subsection{Scenario}
\emph{The TMF Hope is a standard freight ship running cargo between planets. There’s no reason something should have gone wrong with their last standard hyper-jump. But apparently something did, because now the ship seems to be caught between several dimensions, and the ship is being pulled apart. If you could all just work together, I’m sure you could fix this…}

This game focuses on the \pNew{}, a civilian vessel in a near future sci-fi world where long distance space travel is accomplished via “jumps” though hyperspace. The game begins with the ship stuck somehow, in the middle of one such jump. Such a phenomenon is unknown to modern science, and the crew must race against the clock to reverse whatever happened and save the ship and themselves. In this liminal space, communication and access are limited by the ship and crew having been split into several “dimensions” and their overlapping spaces.

\subsection{Logistics}

This game is played in a Discord Server. Players will interact with each other (in character) over voice channels that represent rooms on the ship. They will interact with the environment through associated text channels. Moving from one space to another requires switching both text and voice channels.

\paragraph{Game Times:} Game runs from \textbf{12:00 pm - 4 pm (PST)} on \textbf{Saturday, Jan 30th}. Players are expected to log on to the``TMF Hope'' Discord server and connect to the ``OOC Channel'' voice channel at that time.  Surviving PCs are expected to be in-game on the server for the entirety.  Game may end early.  Wrapup will immediately follow the end of game. \textbf{If you encounter technical difficulties, please CALL the GMs and let us know. Acata: 650-690-5628.}  

\paragraph{Game Spaces:} This game is hosted on a Discord server. You should have received an invite to the server along with your casting. If you did not, please get in touch with us as soon as possible. You will need both the ability to type in the text channels, and connect to the voice channels. Video is entirely optional and this game will not suffer from some or all players using voice only.

The Discord server contains an OOC category, with several OOC channels for pre and post game chatter. There is also an ``announcements'' channel, that will be used by the GMs before and \textbf{during} game to make game-wide announcements. This is all you will be able to see when you first join. Once you join the server and the GM assigns you your role, you will be able to see the game spaces accessible to your character. You will see 1``Private Channel'' matching your character role, and 2 of the ship spaces (each comprised of 1 text channel and 1 voice channel, grouped together and named the same).

The full list of game spaces that comprise the \pNew{} (You will only ever have access to 2 game spaces at a time, in addition to your private channel.)

\begin{multicols}{2}
\textbf{Primary Dimensions}
\begin{itemize}
  \item The Bridge (Yellow)
  \item The Science Lab (Cyan)
  \item The Medical Bay (Magenta)
\end{itemize}
\vfill\null
\columnbreak
\textbf{Secondary Dimensions}
\begin{itemize}
  \item The Gym (Green)
  \item The Cafeteria (Blue)
  \item The Recreation Room (Red)
  \item The Storage Bays (Black)
\end{itemize}
\end{multicols}

%

\clearpage
\section{Game Style}
\subsection{Secrets and Powers}
This game is Secrets and Powers style, inspired by such games as \emph{Krazny Octobyr} and \emph{The HMS Dauntless} from the MIT Assassin’s Guild, and \emph{The Neptune Ball} and \emph{Conflux} from the Stanford Gaming Society. This type of game often includes multiple mutually exclusive goals, that lead to character vs character conflict. This game does not rely on player to player negotiation to generate and resolve plots. Instead, the pre-written characters are already enmeshed in a web of plots with potential resolutions that characters are negotiating for. Players will get the most out of the game by embracing their character, playing to discover the world created by the game writers, and reacting in character to new revelations from other PCs. You may not come out of this game having accomplished everything your character wanted. 

While characters are often playing to win, players should look to the best stories, and embrace the most narratively impactful moment to have their character back down, admit a failure, spill a secret, or change their mind about something, all with confidence that other characters will care. You may also find another character revealing your secret without any input on your part, giving you the opportunity to react to that, without having had to engineer the reveal yourself. This allows players to lean in to characters who want to keep a secret hidden.

This game also contains significant interactions between PCs and the environment. These interactions are regulated through predefined mechanics that allow players to interact with elements of the environment without waiting for a GM to narrate something or make a ruling. This generally leads to less sitting around waiting out of character, and more time playing and interacting in character. Do not invent items or solutions to a problem that already has a mechanic. If a door says it needs a key to open, you cannot declare that you beat the door down, nor can you claim to have the key if you don't actually have the key item.  If there is no mechanic, plots should be resolved via \textbf{in character} discussion (aka: roleplaying).

\subsection{Calibration and Safety Mechanics}
\textbf{Players are more important than the game}. In ``\pNew{},'' this adage must be applied preemptively, not just when someone gets activated. Secrets and Powers games are fragile. While they can survive a character not being present, the game is significantly the poorer for it. This is due to the design concept that any character the game could run without should either be cut before casting, or rewritten to be more integral. This makes it way less likely that a player will be bored or feel left out, but it does pose important considerations for safety.

Calibration is handled primarily through casting. Since players are expected to play to the pre-written character they are given, players can be confident that the provided content warnings for game are comprehensive.  In order for the GM to cast players well, players must be as honest as they can be when filling out the casting survey. It is possible that ultimately the player and GM determine that this game is not a good fit for a particular player.

In order to  balance the needs of the game and other players with self-care needs, players are strongly encouraged to the following:
\begin{itemize}
  \item Reflect on the game scenario and the listed content warnings (see below) before applying. If you have reasons to believe you’ll be activated by game content, consider skipping this game.
  \item Reflect on your emotional reserves right now, and what actions you can take to prepare care for yourself before game to boost your resiliency (i.e.: making sure you get a full night's sleep before game). Consider what preparations you can make for self care after game to replenish (i.e.: ordering your favorite food, or having a stuffed animal nearby for snuggles). This can be helpful in case you get unexpectedly activated.
  \item Reflect on the safety mechanics for this game and how you can use them to protect yourself and create space to check in with yourself. If you need to, leave the game. The door is always open, and if it’s what you need to take care of yourself, do it. But if a less nuclear option feels accessible, we encourage you to try that one first.
\end{itemize}

This is only a game.  Everyone involved should act with courtesy, sportsmanship, patience, and taste.  The GMs may expel anyone they believe to be violating the spirit of the rules or the game.  Emotions may run high.  If you think things are crossing the line from game to reality too much, or if you are just getting too stressed, feel free to disconnect and take a quick break. Always, play safely, then play to have fun. Real violence is unacceptable. Game action should cause no real-world damage, either to people or property.  If something dangerous is happening, call a halt.  Stay in control, use common sense, and do not endanger yourself or others.

We will be using the following safety mechanics in game:
\begin{itemize}
  \item ``Brake'' (from ``Cut and Brake'')  - Use this to re-calibate the auditory experience of an interaction for a player’s comfort. I.e.: Ask a player to speak more quietly, even though their character is still yelling. This is not a negotiation - the person being asked to ``brake'' should immediately comply for the comfort and safety of their fellow players. Example: 	*Fist on Head in OOC gesture*. ``Hey Bob, brake. Can you not shout please?''
  \item ``Time Out/Time In'' - Use this to step away from game for up to about 5 minutes. Call ``Time Out,'' then drop off of the voice channels, and set yourself to ``invisible'' in Discord. Then take whatever other actions you need to distance yourself from game for a bit (i.e: Standing up and stretching, going into another room, etc). Use this to recenter on your needs and experiences and evaluate if you need to make any changes to how you play or are interacting, or need to ask others to do so. Please try to use ``time out'' before ``open door''. If someone calls ``Time Out'' around you; leave them alone. Let them disengage without further fanfare. Return to game by saying ``Time In''
  \item ``Open Door'' - If you need to leave the game and not return, either for self care or emergency reasons, please do so. If possible, let a GM know so we don't worry about you and try to reach you if you just want to be alone or need to deal with something else.
\end{itemize}

\subsection{The Casting Process}
The GMs have done their best to cast you to a pre-written character that you will enjoy, but it is not always possible to give everyone exactly the character they had in mind - it might not exist. If you receive your character and find that there is something that you absolutely can't play, please let us know as soon as possible. We can recast that character to someone on the wait-list. If you get a character that isn't what you expected, we encourage you to try it out - you might enjoy it. 

In any case, we encourage you to reflect on your casting app after the fact, and see whether there are things you want to add, remove, or change in future apps for other games. You GMs can only cast as well as you let them - they only know what you tell them.

\subsection{Content Warnings}
The following content warning applies to all of game. We recommend you skip this game if you don’t want to play a game containing these things: Potential for character death; Risk to health and safety.

The following content warnings apply to parts of game. It is possible for GMs to cast you in roles away from these topics if you include it in your application that you wish to avoid the topic. (warning: If you need to avoid \textbf{all} of these things, this may not be the game for you):
\begin{multicols}{2}
\begin{itemize}
  \item Ex-millitary
  \item Dishonorable discharge
  \item Decisions around life and death for others
  \item Romance
  \item Strong animosity with another character
  \item PTSD from service in combat / interacting with someone with PTSD
  \item Abelism
  \item Ageism
\end{itemize}
\end{multicols}

FOnce you get your character, please stay within the bounds outlined by that sheet and the game content in general. Do not invent edgelord backstory elements, or introduce new plot points. This is important for the integrity of game. If you invent something new and abandon content written for the character, you leave the other characters involved in that content out in the cold. This is also crucial for the safety of your fellow players since they didn’t get a chance to opt into or out of the new content.

\section{Character Packets}

Information on your your part as a \textbf{player-character} (\textbf{PC}) and the world they live in will be presented in a series of documents collectively referred to as your ``character packet''. The term comes from in person games where the character information would be printed out and handed to you in a manila envelope - a literal packet. You are expected to be familiar with the contents of your character packet before game. We recommend you keep the various documents handy during the game, possibly on a second screen, or printed out.  If you are missing something or find you've been given something which doesn't seem to belong to you, tell one of the GMs.  Character packets are confidential; please do not share any information you got on these documents with anyone else - in or out of character. You may think two characters should know the same things, but you may inadvertently ruin a major surprise for them.\\

Your Character Packet should contain:

\paragraph{Character Sheet:} Your character sheet describes who you are and what you are up to.  It contains a list goals and contacts at the bottom, along with any bluesheets and greensheets you should have. Do not show or read your character sheet to other players.

\paragraph{Bluesheets:} A bluesheet describes information common to members of a group.  When in conflict, character sheet information overrides bluesheet information.  Do not show or read a bluesheet to other players. (These are called ``bluesheets'' because they would traditionally be printed on blue paper.)

\paragraph{Greensheets:} A greensheet describes and expands abilities, mechanics, or in-game knowledge.  Do not show or read a greensheet to other players. (These are called ``greensheets'' because they would traditionally be printed on green paper.)

\subsection{Discord-based Character Information}
These parts of your character exist only in the Discord server. They won't be listed on your character sheet. Use the ``?inventory'' command in your private channel (see below) to get a list of what your character has. 

\paragraph{Abilities:} Some characters have abilities that represent small mechanics, or something your character can do, or is particularly good or bad at. You will use text-based commands to use your abilities.

\paragraph{Memory/Event Trigers:} A memory/event trigger represents your character remembering, or learning something, or something happening to your character. When something or someone tells you to ``trigger your ``Hope'' packet'', you will go to your private channel, and use the bot command ``?Trigger Hope'' \textbf{You will then receive a PRIVATE MESSAGE from the LARP Bot (not a message on the server) with the contents of the packet.} 

Some memory packets will give you just a piece of information, suggest something, or change a stat. Others may provie information along a line of investigation and proide a new trigger for when the next task is complete. (These replace research notebooks in this game.) Do not take game action based on an unopened trigger.  Do not show or read a memory packet to other players.

\paragraph{Items:} In-game items may be transferred from character to character, as long as the two characters are in the same game space (text channel and voice channel). Some characters start with one or more items in their posession. Some items are \textbf{bulky} this means that you must dedicate that many hands to carrying it. Items can have a bulkiness of 0 (not bulky), 1, or 2. Unless you know otherwise (aka: you have a mechanic that says so), you can never carry more than 2 bulkiness-worth of items as you only have 2 hands. If you need 3 bulkiness worth of items to accomplish something, you will need someone else to help you by carrying the 3rd item.

Items cannot be dropped on the ground. Once you pick one up, you can only pass them to other characters, or put them in containers.

\paragraph{Discord Pinned Message:} In your private channel, you will find a pinned messages, with your stats listed. This is to give you easy reference to them during game. NOTE: your stats may change during game. If you wish to have that message updated, tag a GM in your private channel with the stat and its new value.

\subsection{Reality and Game Reality}

There is a big difference between reality and game reality.  Players must treat each other with courtesy and explain to each other what their characters perceive in confusing situations; e.g.\ ``My character's hands are covered in blood,'' an \textbf{out-of-game} statement.  Characters are under no such restrictions, and may do what it takes to further their goals; e.g.\ ``Uh, hi Bob.  Just got back from the butcher shop,'' an \textbf{in-game} statement.

\textbf{Metagaming} is inferring in-game knowledge that is inappropriate for your character from out-of-game information.  Do your best to not metagame and especially to prevent the risk of metagaming. Be your own harshest critic.

\paragraph{Halts:} A halt pauses game action.  To call one, say ``game halt'' in a clear and audible voice; all players in your voice channel should hear you. End a halt by saying ``three, two, one, resume.''  Call a halt for one of only three reasons: because a rule instructs you to, for safety and similar out-of-game issues, or to pause game and wait for a GM (which you should normally not need to do, but a few mechanics may tell you to ``@ the GM role'' aka type ``@GM: $<$Message$>$'' into the text channel).

\paragraph{Not-Here:} You may go not-here by setting yourself to ``invisible'' in Discord. Go ``not-here'' only if a rule instructs you to, or to take a ``Time Out''.

\paragraph{Observers:} An observer is someone not playing the game who has agreed to watch.  They will get an ``observer'' role to make them more visually apparent on the server.  Observers have traditionally been called ``ghosts.''  They should stay out of the way; you can always ask an observer to leave.  If a friend who is not playing wants to observe game, send them to the GMs.

\paragraph{Mechanics:} Many actions your character can take, such as, talking with other characters, are represented by you doing them.  Others, like combat, are performed via
abstract mechanics, which are described in ability cards, greensheets, and rules.  The abstract information for mechanics (like CR) may not be discussed in-game.  If you want to do something special for which there is no mechanic, ask a GM.

Become familiar with your mechanics before game starts, especially those which occur under time-pressure (like combat).  Game action will not stop for memory packets, greensheets, or such.

A \textbf{kludge} (and derivative forms like ``kludge-ite'') is something impervious to logic and cleverness, usually for game-balance.  You can't affect a kludge without a specified mechanic.

\subsection{Basic Strategy}

Make sure you understand the rules.  If you are completely confused, get a GM who will try to help you out.  Make sure you know enough about your character to role-play them  when you start talking to other people.  Read through your entire packet a couple of times, and skim through it again right before game starts.  If you don't know something about your character, ask a GM.

As a character, your first priority should be to open lines of communication.  Contact people, show up at meetings, and chat.  Try to be easy to get in touch with.  Ask people questions on relevant subjects. They'll probably lie, but you may find something out. There are no guarantees that you can trust anyone, but since cooperation is the key to accomplishing things, you will be forced to trust people anyway.  The most trustworthy people are probably those who need you.

\section{The Game World}

\paragraph{Signs:} Some locations and other game materials are represented by ``signs''. Each text channel in the discord (except your private one) has signs associated with it, that convey information about the world you are in.  You may read any signs and must follow any rules printed on them.

\paragraph{Containers:} Containers are a special type of sign. These signs have the ability to store items. By reading the description on the sign, you can get a sense of what might be in the container. If you meet the conditions for doing so, you can search and take, or add items to a container. If no conditions are listed, anyone may do this. Most containers will take 30 seconds to search, this means you won't see what you find until 30 seconds after you elect to search the container. \textbf{Stay in the same place - text and voice - until you get your results.} This time is representing the time it takes for your character to search the space - it wouldn't make sense for you to wander off to a different location, and then wander back to see what you found, so please don't do it.

\paragraph{Character Bodies:} Character bodies cannot be moved, hidden, or destroyed in any way. (This is a kludge)

\subsection{Searching, Stashing, and Stealing}
To search a place, search it using the ``?search'' command. To search a container, search it using the ``?searchcontainer $<$container name$>$ '' command. (see details in appendix).

\paragraph{People:} All searches of characters or their belongings are conducted via player dialogue and discord text.  Someone must be willing or unable to resist for you to search them.  You need at least one free hand to search someone. A search will reveals all in-game items. Once someone consents to the search, or they are unable to resist, have them type ``?giveall $<$@user$>$'' with you as the user. This will hand over all of their items to you.

\section{Violence, Damage, and Death}

\subsection{Health States}

Characters have five possible states, concerning health and damage. When you are \textbf{fine}, you may act freely.  When you are \textbf{restrained}, you are helpless and may do nothing but talk.  When you are \textbf{knocked out}, you will wake up in five minutes.  When you are \textbf{wounded}, you are unconscious, bleeding, and will die in five minutes.  When \textbf{dead}, you are dead.

When knocked out or wounded, disconnect from the voice channel and start a 5 minute timer. Whenever someone joins the voice channel, @ their name or role, and tell them you are lying on the floor knocked out or wounded. I.E.: ``@Bob: I'm wounded on the floor.''

Dead men tell no tales. If your character dies, do not give out any information about your character or death to any players.  You may remain on the scene to play the part of your corpse. We encourage you to do so for a few minutes. Describe obvious information to onlookers (``I have a gunshot wound in my back'').  When you go to leave, \textbf{tell a GM} so we can add a container describing the body and to hold any items you had. If your death becomes generally known to the other characters, you may be able to become an observer.  Until the game is over, you may not convey game information to any player.


\subsection{Martial Combat}

%% intro
All characters have a \textbf{Combat Rating} (\textbf{CR}) stat.  This represents your basic skill in martial combat; you use the same number for attacking and defending (unless you know otherwise).  Someone with a CR of one can't fight very well.  Someone with a CR of three is a skilled fighter. When using this stat, you may pull your punches by using a lower number.

%% offense
To martial-attack someone, clearly state your target, your attack and your CR (``Captain Jones, \aKnockOut{} 2'', ``Mary, \aWound{} 2'', etc.). You must be in the same voice channel with the person you are attacking. Your first attack must resolve before you make another; otherwise, you may act freely (aka you may talk).  If an ally directs \textbf{\aAssist{}} at you after you attack, you may, within 2 seconds, restate your attack with the \aAssist{}'s CR added (``Captain Jones \aWound{} 3'', ``Mary \aAssist{} 2'', ``Captain Jones \aWound{} 5''). You may ignore an \aAssist{}. \textbf{You may not assist defense, only attacks}.

Knock Out and Wound attacks do not require any items.

\paragraph{Rope:} Rope is \textbf{not} freely available. You will need to find an item that says it can be used as rope. To tie someone up, they must be either willing, helpless (unconscious or wounded), or you must win a martial-attack against them.  If you get tied up with rope, you become restrained. You are unable to change game spaces (channels) on your own, but can be moved by someone else. Moving someone who is restrained requires 1 hand free (you cannot carry an object with bulkiness 2 at the same time).  If you are conscious and left alone for 3 minutes (no one else in the same game space with you).

%% defense
When martial-attacked, resolve by comparing the attack against your CR.  If your CR is lower, take the effects; else, say ``\textbf{resist}'' and the attack has no effect. \textbf{Do not state your CR when you resist. Just say ``resist''}.  If you neither say ``resist'' nor state your own attack within two seconds of the incant's end, you are surprised and the attack just works.  Serial attacks don't prevent simple actions (talking) in-between.  Resolve all attacks alone, in the order they occur; choose the order if it is unclear. 

\textbf{You may not assist defense, only attacks}. If you have an ally but your CR is not high enough to resist an attack, your best option might be to attempt a counter attack. At best this will leave you and your attacker knocked out or wounded, and your ally can heal you up (or go through your pockets.)

There are no stealth attacks of any kind in this game.

\paragraph{Martial Attack Abilities:} You should assume that every character has \aKnockOut{}, \aWound{},  \aAssist{}, and \aRestrain{}.  Other attack abilities may exist. Assume you have the ability to use all of these attacks unless you know otherwise (i.e.: your character sheet says ``you may not use \textbf{knock out} attacks.'')\nopagebreak

\subsection{Killing Blow}

Killing blows are a the only ``n-count'' mechanic in this game (it is also the only interrupt-able mechanic in game). If your character wishes to kill another character, but don't feel like waiting for them to bleed out from a wound attack (or don't want anyone to find them and heal them), you must use the ``Killing Blow'' mechanic. \textbf{You must have both hands free, aka not holding any bulky items to perform a killing blow.} If you have a willing or helpless (restrained, knocked out, wounded) victim, you may begin an n-count of 10 by:
\begin{enumerate}
  \item Type into the text channel: ``Captain Jones Killing blow 0'' (Indicating who you are targeting with the mechanic)
  \item Speak in the voice channel, and a reasonable speed and volume: ``Killing blow 1,'' ``Killing blow 2''\ldots up to ``Killing blow 9.'' 
  \item Type into the text channel: ``Captain Jones Killing Blow 10.''
\end{enumerate}  

If you reach 10, then your target is dead. At any point before you reach 10 however, someone else may say ``I stop you,'' or engage you in combat. This interrupts you, and you will have to start your count over again. It is necessary to start and end in the text channel in order to give someone who is just moving into that game space and has not yet connected to voice a chance to realize what is going on. It also gives an unconscious or wounded victim knowledge of what is happening (since they are supposed to disconnect from the voice channel.) We won't make you type the whole thing as it can be tedious.

\subsection{Healing}
Only characters with the ``First Aid'' ability can heal. Otherwise you cannot heal people. Dead characters cannot be healed.

\section{TMF Hope Specific Changes}
This section is a recap of the changes specific to this game from other MIT Assassins Guild style games played at Stanford and elsewhere in the recent past. There are also several new and important game-wide mechanics. Familiarize yourself with them before game.

\textbf{Items cannot be dropped on the ground.} Once you pick one up, you can only pass them to other characters, or put them in containers. You can only carry 2 hands bulky worth of items (1 item that is 2 hands bulky, or 2 that are 1 hand bulky each).

\subsection{Safety}
Your safety tools for this game are:
\begin{itemize}
  \item ``$<$Player Name$>$ Brake''
  \item ``Time Out" / ``Time In''
  \item Open Door Policy - please tell a GM, if you can, if you are leaving permanently.
\end{itemize}

\subsection{Combat}
The game specific changes to Darkwater combat are:
\begin{itemize}
  \item You need an item that can serve as rope to make a ``restrain'' attack.
  \item You do not need an item to make ``knock out'' or ``wound'' attacks.
  \item Killing blows must be on a helpless or willing target. They require a 10-count and must be posted in the text channel as well as being spoken in the voice channel (see above)
  \item There is no ranged combat in this game.
  \item There is no waylay mechanic (stealth combat) in this game. 
\end{itemize}

\textbf{Reminder: You cannot assist defends.}

\subsection{Discord}
Reminders about the Discord Server:
\begin{enumerate}
  \item You must change voice channels and text channels together to change game spaces. The exception is moving to your private text channel to interact with a mechanic. Make sure you come back to the same text channel as the voice  channel you are still in.
  \item You can only pass items between characters currently in the same game space as each other.
  \item If you have a memory packet that you are supposed to trigger at the start of game, don't forget to do that (check your memory packets in the Discord with ``?inventory''. It is the start of a research chain that is important to your character.
  \item You can always use ``@GM'' in the matching text channel to call a GM to your text/voice channel.
\end{enumerate}

\section{Closing Notes}

These rules are imperfect.  The GMs may violate the letter of the rules to preserve the spirit.  We hope these rules are reasonably clear, but if you have any doubts about your interpretation, talk it over with us in advance.  We should also add, as much as we hate to admit it, we GMs are human: when all of our carefully laid plans are going haywire, we may lose our cool.  The best way to deal with people is remaining calm and friendly, especially when everyone is tired and hungry.

We hope you have lots of fun.  Good luck.

\clearpage
\appendix
\section{Discord Commands For Players}
The ``Playground'' channel in the OOC category of Discord will give you a chance to practice these commands before game. Take advantage of it! The more familiar you are with the commands, the more natural it will all feel.

\begin{tabularx}{\textwidth}{|>{\raggedright\arraybackslash} p{3cm}|>{\raggedright\arraybackslash} p{3cm}|X|X|}
 \hline
 
\multicolumn{4}{|c|}{\textbf{Interacting with the World}} \\
\hline
\emph{Command} & \emph{Example} & \emph{Effect} & \emph{When to use}\\
    \hline
   ?search    & ?search    & Returns a list of signs and containers in the area. &  Use in the Game Space channel you wish to explore. \\
    \hline
   ?readsign $<$sign name$>$   & ?readsign A Chalkboard    & Read a Sign for a description of something in the environment. &   Use in the Game Space channel where the sign is located.\\
    \hline
   ?changesign $<$current sign name$>$ . $<$new sign name$>$ & ?changesign $<$A Chalkboard$>$ . $<$A secret video screen$>$   & Switch one sign out with another one. The sign being changed out will tell you the name of the sign to be changed in. &   Use in a Game Space text channel when a sign prompts you to switch it.\\
    \hline
  ?searchcontainer $<$container name$>$   & ?searchcontainer A mysterious box    & Use to read a container, optionally search for what items are in it, and optionally to take an item from the container. &   Use in the Game Space channel where the container is located.\\
    \hline\hline
\multicolumn{4}{|c|}{\textbf{Items}} \\
\hline
\emph{Command} & \emph{Example} & \emph{Effect} & \emph{When to use}\\
    \hline
  ?inventory   & ?inventory    & Check your own inventory for items, abilities, and mem-packets (each listed by name) &  Use only in your private channel as it prints secret information.\\
    \hline
  ?listitem $<$item name$>$   & ?listitem banana    & Gives you the name, description, and bulkiness of the item. &   Use in your private channel to see the details of an item.\\
    \hline
  ?give $<$@user$>$ $<$item name$>$   & ?give @Bob banana    & Transfer an item in your inventory to the tagged user's inventory.\textbf{You may only pass items if you are both in the same game space.} &  Use in a Game Space channel with the character you wish to pass the item to.\\
    \hline
		?drop $<$item name$>$ . $<$container name$>$   & ?drop banana . A mysterious box   & Move an item from your inventory to a container &   Use in the Game Space channel where the container is.\\
    \hline\hline
\multicolumn{4}{|c|}{\textbf{Abilities}} \\
\hline
\emph{Command} & \emph{Example} & \emph{Effect} & \emph{When to use}\\
    \hline
  ?abilities   & ?abilities    & list your abilities, their description and their effects &   Use only in your private channel as it prints secret information.\\
    \hline
  ?useability $<$@user (optional)$>$ $<$ability name$>$   & ?useability @Bob Sleep -OR- ?useability Make Art  & Use an ability. It will print the ability effect and tag the target if necessary. &   Use in the Game Space channel where you wish to execute the effect. If the ability requires a target, you must be in the same chanel.\\
    \hline\hline
\multicolumn{4}{|c|}{\textbf{Memory Packets, Event Triggers, and Research Chains}} \\
\hline
\emph{Command} & \emph{Example} & \emph{Effect} & \emph{When to use}\\
    \hline
  ?trigger $<$memory packet name$>$ & ?trigger A packet & Opens up a memory packet. Only use this when you have met the requirements listed in the trigger. &  Use in your private channel. You will receive your response via Private Message from the Bot.\\
    \hline
\end{tabularx}

\end{document}
